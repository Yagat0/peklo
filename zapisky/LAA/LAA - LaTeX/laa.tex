\documentclass[12pt, a4paper]{scrartcl}
\usepackage{graphicx}
\usepackage{amsmath}
\usepackage{dsfont}
\usepackage{multicol}

\title{Lineární algebra}

\begin{document}
    \maketitle
    \tableofcontents

    \section{Komplexní čísla}
    Komplexní čísla jsou rozšířením oboru reálných čísel.
    V algebraickém tvaru se zapisují jako $a + bi$,
    kde $a,b \in \mathds{R}$. $a$ nazýváme reálnou a $b$ imaginární částí. 

    \subsection{Operace s komplexními čísly}
    Sčítání, odčítání a násobení funguje tak, jak bychom čekali.
    \[z = 1 + 4i\]
    \[u = -2 + 3i\]
    \[z + u=(1 + 4i) + (-2 + 3i) = -1 + 7i\]
    \[z - u=(1 + 4i) - (-2 + 3i) = 3 + i\]
    \[z * u=(1 + 4i) * (-2 + 3i) = -14 - 5i\]

    Dělení je speciální případ. Pro dělení dvou 
    komplexních čísel rozšíříme zlomek komplexně sdruženým číslem jmenovatele,
    které získáme obrácením prostředního znaménka.
    \[\frac{u}{z} = \frac{u}{z} * \frac{\bar{z}}{\bar{z}} =
    \frac{-2 + 3i}{1 + 4i} * \frac{1 - 4i}{1 - 4i}=\frac{-2 + 3i + 8i - 12i^2}{1 - 4i + 4i - 16i^2} =
    \frac{10 + 11i}{17} = \frac{10}{17} + \frac{11}{17}i\]

    \subsection{Goniometrický tvar}
    K zapsání goniometrického tvaru nám postačí absolutní hodnota komplexního
    čísla a úhel, který svírá s osou $x$.

    \[z = |z| * (\cos \alpha + i \sin \alpha)\]

    Příklad 1: $z = -1$ \\
    Goniometrický tvar: $z = 1 * (\cos \pi)$

    Příklad 2: $|z| = \sqrt{2}; \alpha = \frac{7}{4} \pi$ \\
    Goniometrický tvar: $z = \sqrt{2} * (\cos \frac{7 \pi}{4} + i * \sin \frac{7 \pi}{4})$
    
    \subsubsection{Moivreova věta}
    Moivreova věta nám slouží k umocňování a odmocňování komplexních čísel.
    \[z^n = |z|^n * (\cos \frac{\alpha + 2k \pi}{n} + i \sin \frac{\alpha + 2k \pi}{n})\]
    \[\sqrt[n]{z} = \sqrt[n]{|z|} * (\cos \frac{\alpha + 2k \pi}{n} + i \sin \frac{\alpha + 2k \pi}{n})\]

    Příklad:
    \begin{align*}
        \sqrt[3]{-1} &= \cos \frac{\pi + 2k \pi}{3} + i \sin \frac{\pi + 2k \pi}{3}, \quad k \in \{0,1,2\} \\
        k = 0: &\quad \sqrt[3]{-1} = \cos \frac{\pi}{3} + i \sin \frac{\pi}{3} = \frac{1}{2} + i \frac{\sqrt{3}}{2} \\ 
        k = 1: &\quad \sqrt[3]{-1} = \cos \frac{3\pi}{3} + i \sin \frac{3\pi}{3} = -1 \\ 
        k = 2: &\quad \sqrt[3]{-1} = \cos \frac{5\pi}{3} + i \sin \frac{5\pi}{3} = \frac{1}{2} - i \frac{\sqrt{3}}{2} 
    \end{align*}
    \section{Analytická geometrie}
    \subsection{Vektor}
    Pro $n \in \mathds{N}$ se zapisuje jako $\vec{u} = (u_1, u_2, \dots, u_n)$
    
    \subsubsection{Vektorový součin}
    Je definována pouze na dvou vektorech v $\mathds{R}^3$. 
    Výsledkem je vektor na ně kolmý.
    \[\vec{u} = (u_1, u_2, u_3)\]
    \[\vec{v} = (v_1, v_2, v_3)\]
    \begin{equation*}
        \vec{u}\times \vec{v} = 
        \begin{pmatrix}
            u_2 v_3 - u_3 v_2 \\
            u_3 v_1 - u_1 v_3 \\
            u_1 v_2 - u_2 v_1
        \end{pmatrix}
    \end{equation*}

    \textbf{Pozor!} $\vec{u}\times \vec{v} \neq \vec{v}\times \vec{u}$.
    Oba vektory jsou kolmé, ale každý na opačnou stranu.

    \subsection{Přímka}
    Parametrické rovnice přímky se zapisují jako $X = A + \vec{u} t$, 
    kde $A = [a_1, a_2, a_3]; \vec{u} = (u_1, u_2, u_3)$.
    \[x = a_1 + u_1 t\]
    \[y = a_2 + u_2 t\]
    \[z = a_3 + u_3 t\]

    Příklad: Nalezněte průsečík X přímek $p$ a $q$
    \begin{multicols}{2}
        \begin{equation*}
            p:\;\begin{aligned}
            x &= 1 + t \\
            y &= -2 + t \\
            z &= -1 - t
            \end{aligned}
        \end{equation*}

        \begin{equation*}
            q:\;\begin{aligned}
            x &= 0 + s \\
            y &= 1 - s \\
            z &= -2
            \end{aligned}
        \end{equation*}
    \end{multicols}

    \begin{multicols}{4}
        \begin{equation*}
            \begin{aligned}
                1 + t &= 0 + s \\
                -2 + t &= 1 - s \\
                -1 - t &= -2
            \end{aligned}
        \end{equation*}

        \begin{equation*}
            \begin{aligned}
                t - s &= -1 \\
                t + s &= 3 \\
                t &= 1
            \end{aligned}
        \end{equation*}

        \begin{equation*}
            \begin{aligned}
                1 - s &= -1 \\
                1 + s &= 3 \\
                t &=1
            \end{aligned}
        \end{equation*}

        \begin{equation*}
            \begin{aligned}
                s &= 2 \\
                s &= 2 \\
                t &= 1
            \end{aligned}
        \end{equation*}
    \end{multicols}

    Jelikož se parametry $s$ v první i druhé rovnici shodují, tak mají přímky průsečík. 
    Souřadnice bodu X získáme dosazením parametru $t$ do parametrických rovnic přímky p
    \begin{equation*}
        X:\;\begin{aligned}
            x &= 1 + t = 1 + 1 = 2 \\
            y &= -2 + t = -2 + 1 = -1 \\
            z &= -1 - t = -1 - 1 = -2
        \end{aligned}
    \end{equation*}

    \subsection{Rovina}
    Parametrické rovnice roviny se zapisují jako $X = A + t \vec{u} + s \vec{v}$, 
    kde $A = [a_1, a_2, a_3]; \vec{u} = (u_1, u_2, u_3); \vec{v} = (v_1, v_2, v_3)$.
    \[x = a_1 + u_1 t + v_1 s\]
    \[y = a_2 + u_2 t + v_2 s\]
    \[z = a_3 + u_3 t + v_3 s\]
\end{document}