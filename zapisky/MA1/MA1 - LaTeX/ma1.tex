\documentclass[12pt, a4paper]{scrartcl}
\usepackage{graphicx}
\usepackage{dsfont}
\usepackage{amsmath}
\usepackage{amssymb}

\title{Matematická analýza 1}
\date{2025/2026}

\begin{document}
    \maketitle

    \tableofcontents

    \section{Výroková logika}
    \subsection{Výrok}
    Výrok -- oznamovací věta, u které má smysl se bavit o pravdivosti, 
    avšak pravdivost nemusí být zjistitelná. O výroku, pro který lze 
    o pravdivosti rozhodnout a zároveň je pravdivý říkáme, že je dokazatelný.

    Příklady:
    \begin{itemize}
        \item{Venku prší. -- je výrok.}
        \item{$1 + 1 = 2$ -- je výrok.}
        \item{Běž ven. -- není výrok.}
        \item{$x + 2 = 3$ -- není výrok.}   
    \end{itemize}

    \subsection{Negace výroku}
    Negace výroku -- má opačnou pravdivostní hodnotu.
    ``Není pravda, že A.`` Zapisuje se jako $\rceil A$
    %Ekšly je tohle asi nejsložitější věc z toho všeho, 
    %ale nevěnovali jsme tomu pozornost lmao, takže se k tomu někdy vrátím.
    
    \subsection{Složené výroky}
    Složené výroky -- výroky lze spojovat do složených výroků pomocí logických spojek.

    \begin{itemize}
        \item{Konjukce ($\wedge$) -- ``A a zároveň B.``}
        \item{Disjukce ($\vee$) -- ``A nebo B.``}
        \item{Implikace ($\Rightarrow$) -- ``Pokud A, potom B.`` ``Je-li A, potom B.``}
        \item{Ekvivalence ($\Leftrightarrow$) -- ``A právě tehdy, když B.`` (Pozn.: (A $\Rightarrow$ B) $\wedge$ (A $\Rightarrow$ B))}
    \end{itemize}

    Tabulka pravdivostních hodnot:
    \begin{tabular}{|c|c|c|c|c|c|c|}
        \hline
        A & B & $\rceil$A & A $\wedge$ B & A $\vee$ B & A $\Rightarrow$ B & A $\Leftrightarrow$ B \\
        \hline
        1 & 1 & 0 & 1 & 1 & 1 & 1 \\
        \hline
        1 & 0 & 0 & 0 & 1 & 0 & 0 \\
        \hline
        0 & 1 & 1 & 0 & 1 & 1 & 0 \\
        \hline
        0 & 0 & 1 & 0 & 0 & 1 & 1 \\
        \hline
    \end{tabular}

    \subsubsection{Pravdivost implikace}
    Implikace $A \Rightarrow B$ je pravdivá vždy, když platí předpoklad $A$ a současně i závěr $B$, 
    nebo když předpoklad $A$ neplatí.  

    Například (Pozn.: $m|n$ značí ``$n$ je dělitelné $m$``.):
    \[ 6|2 \Rightarrow 3|2\]
    Jestliže 6 dělí 2, pak 3 dělí 2. Tato implikace je pravdivá, 
    protože podmínka ``6 dělí 2`` není splněna. Implikace tedy nic neslibuje, pokud předpoklad neplatí.

    Za zmínku stojí i související pojmy:
    \begin{itemize}
        \item $\rceil A \Rightarrow \rceil B$ je \textbf{obměněná implikace}.
        \item $B \Rightarrow A$ se nazývá \textbf{obrácená implikace}.
    \end{itemize}

    \subsubsection{Vztahy v implikaci}
    Implikace $A \Rightarrow B$. A je \textbf{dostačující podmínka} pro B. 
    B je \textbf{nutná podmínka} pro A.  

    Ukážeme si to na příkladu.

    \[\forall (n \in \mathds{N}): 6|n \Rightarrow 3|n\]
    Přeloženo: pro každé přirozené $n$ platí: jestliže číslo 6 dělí $n$, pak číslo 3 dělí $n$.  
    Výrok ``$6|n$`` je pouze dostačující podmínkou, protože každé $n$, které je dělitelné třemi, 
    nezaručuje, že je zároveň dělitelné šesti.

    \section{Množiny}
    \begin{itemize}
        \item Omezená zdola -- $\exists\, d \in \mathds{R} \ \forall\, x \in A : x \geq d$
        \item Omezená shora -- $\exists\,  h \in \mathds{R} \ \forall\, x \in A : x \leq h$
        \item Negace om. shora -- $ \forall\,  h \in \mathds{R}\, \exists\, x \in A: x > h$
        \item Omezená shora i zdola -- konjunkce $(\wedge)$ obou zápisů
    \end{itemize}

    Minimum -- ideální dolní hranice (u omezených množin / intervalů)

    Infimum -- dolní hranice, nemusí být součástí množiny / intervalu, ale plní funkci minima

    Maximum -- ideální horní hranice (u omezených množin / intervalů)

    Supremum -- horní hranice, nemusí být součástí množiny / intervalu, ale plní funkci maxima

    Př.: $A = < 1; \infty )$ -- $\exists \min A = \inf A = 1$ a $\sup A = +\infty$, ale $\nexists \max A$

    Př.: $A = (1; \infty )$ -- $\exists \inf A = 1$ a $\sup A = +\infty$ -- ale $\nexists \min A$ ani $\nexists \max A$

    Každá množina má supremum i infimum, ale nemusí mít minimum a maximum

    \section{Cvičení}
    \subsection{1. cvičení 17. 9. 2025}
    \subsubsection{Příklad 1}
    \begin{enumerate}
        \item $\min A = -4 ;\max A = 10 ;\inf A = -4 ;\sup A = 10$
        \item $\min A = \nexists ;\max A = 6 ;\inf A = -2 ;\sup A = 6$
        \item x
        \item $\min A = 2 ;\max A = 4 ;\inf A = 2 ;\sup A = 4$
        \item $\min A = \nexists ;\max A = \nexists ;\inf A = 1 ;\sup A = 5$ Pozn.: zde se narozdíl od celých čísel můžeme příblížit ideální dolní/horní hranici
        \item x 
        \item x 
        \item $\min A = \nexists ;\max A = 1 ;\inf A = -\infty ;\sup A = 1$
        \item $\min A = \nexists ;\max A = \nexists ;\inf A = -\infty ;\sup A = 3$ // $x * (x^2 - x - 6) < 0 = x * (x-3)*(x+2) < 0$ // interval bude vypadat: $(-\infty ; -2) \cup (0; 3)$
    \end{enumerate}
\end{document}