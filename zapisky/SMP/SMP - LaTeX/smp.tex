\documentclass[12pt, a4paper]{scrartcl}
\usepackage{graphicx}
\usepackage{dsfont}
\usepackage{amsmath}
\usepackage{amssymb}

\title{Seminář pro maticový počet}
\date{2025/2026}

\begin{document}
    \maketitle

    \tableofcontents

    \section{Vektory}
    Vektory = orientované úsečky

    \begin{enumerate}
        \item směr
        \item velikost
        \item orientace
    \end{enumerate}

    $\overrightarrow{u} = (u_1, u_2)$

    $\overrightarrow{v} = \overrightarrow{AB}, A[1, 2, -3], B[5, 0, 2]$

    $\overrightarrow{AB} = B - A = (4, -2, 5)$

    Velikost vektoru - $|\overrightarrow{v}| = |\overrightarrow{AB}| = \sqrt{16 + 4 + 25} = \sqrt{45} = 3\sqrt{5}$

    $|\overrightarrow{u}| = \sqrt{u_1^2 + u_2^2}$ ve 2D nebo $|\overrightarrow{u}| = \sqrt{u_1^2 + u_2^2 + u_3^2}$ ve 3D

    Opačný vektor - $\overrightarrow{BA} = -\overrightarrow{AB}$

    Součin čísla a vektoru - $7 * \overrightarrow{u} = (7 * u_1) + (7 * u_2)$

    $|\lambda * \overrightarrow{u}| = |\lambda| * |\overrightarrow{u}|$

    $|\overrightarrow{u} + \overrightarrow{v}| \leq |\overrightarrow{u}| + |\overrightarrow{v}|$

    Př.: $\overrightarrow{u} = (-2, 1, 3), \overrightarrow{v} = (4, -2, 5)$
        \begin{enumerate}
            \item $\overrightarrow{u} + \overrightarrow{v} = (2, -1, 8)$
            \item $\overrightarrow{u} - \overrightarrow{v} = (-6, 3, -2)$
            \item $5 * \overrightarrow{v} = (20, -10, 25)$
            \item $(-3) * \overrightarrow{u} = (6, -3, -9)$
        \end{enumerate}
    
    Střed úsečky - $S = A + \frac{1}{2}\overrightarrow{AB}$

    Př.: $A[1, 1, 1], B[3, 0, -2], C[-3, -1, 7]$ - kdy leží všechny na stejné přímce?

    Tehdy, kdy $\overrightarrow{AB}$ a $\overrightarrow{AC}$ mají stejný směr, tedy: $\exists\, k \in \mathds{R}: \overrightarrow{AC} = k * \overrightarrow{AB}$

    $\overrightarrow{AB} = (2, -1, -3)$

    $\overrightarrow{AC} = (-4, -2, 6)$

    $-4 = k * 2 \rightarrow k = -2;\;-2 = k * (-1) \rightarrow k = 2;\;6 = k * (-3) \rightarrow$ body A, B, C neleží na jedné přímce.

    Př.: $\alpha, \beta \dots$ reálná čísla, aby: $\overrightarrow{u} = (\alpha , 3, -6) a \overrightarrow{v} = (2 , 1, \beta)$ byly lineárně závislé

    $\overrightarrow{u}, \overrightarrow{v} LZ \leftrightarrow \exists\, k \in \mathds{R}:$
    
    $\overrightarrow{v} = k * \overrightarrow{u}$

    $2 = k * \alpha$

    $1 = k * 3$

    $\beta = k * (-6)$

    $k = \frac{1}{3}; 2 = \frac{1}{3} * \alpha \rightarrow \alpha = 2 * 3 = 6$

    $\beta = \frac{1}{3} * (-6) \rightarrow \beta = -2$

    $\overrightarrow{u} = \dots$

    $\overrightarrow{v} = \dots$
\end{document}